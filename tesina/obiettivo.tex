L'obiettivo di questo progetto è di cercare architetture efficaci per predire il risultato delle partite di calcio. Esso si basa sul lavoro svolto da due studenti dell'Università di Stanford \cite{cs230:2020}; per raggiungere tale scopo sono state utilizzate le seguenti librerie:
\begin{itemize}
    \item Tensorflow
    \item Keras
    \item Scikit-Learn
\end{itemize}

% \begin{figure}[h]
%     \centering
%     \includegraphics[scale=0.7]{tesina/img/tensorflow_logo.png}
%     \caption{Logo Tensorflow}
%     \label{fig:my_label}
% \end{figure}
\begin{figure}[h]
    \centering
    \begin{subfigure}{.5\textwidth}
      \centering
      \includegraphics[width=.8\linewidth]{tesina/img/tensorflow_logo.png}
      \caption{Logo Tensorflow}
      \label{fig:sub1}
    \end{subfigure}%
    \begin{subfigure}{.5\textwidth}
      \centering
      \includegraphics[width=.4\linewidth]{tesina/img/Scikit_learn_logo.png}
      \caption{Logo Scikit-learn}
      \label{fig:sub2}
    \end{subfigure}
    \label{fig:test}
\end{figure}
