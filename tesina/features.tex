% \subsection{Dataset di partenza}
Il dataset è stato realizzato analizzando i dati corrispondenti ai risultati delle partite di Premier League [3]
dalla stagione 2008/2009 alla stagione 2018/2019.
Successivamente il lavoro è stato svolto con un dataset caratterizzato da 4180 righe x 78 colonne, 
36 squadre e 40 features (Date, HomeTeam, AwayTeam, FTHG, FTAG, FTR, HTAG, 
B365A, B365D, B365H, BSA, BSD, BSH, BWA, BWD, BWH, GBA, GBD, GBH, IWD, IWD, IWH, LBA, LBD, LBH, PSA, PSD, PSH, SBA,
SBD, SBH, SJA, SJD, SJH, VCA, VCD, VCH, WHA, WHD, WHH), poi utilizzate per la manipolazione ed estrazione di 38 features (refer appendix 2).

Per ogni partita abbiamo accesso al country id, league id, season, stage, date, match api id,
home team id, away team id, home team goal, away team goal, home player X positions,
away player X positions, home player Y positions, away player Y positions, home player ids,
away player ids e le quote delle varie scommesse (B365A, B365D, B365H, BSA, BSD, BSH, BWA, BWD, . . . .)

Per ogni squadra abbiamo accesso al team id, team name, FIFA id, FIFA data 
and FIFA statistics (che includono build up play speed, build up play dribbling, etc.).
Poiché quest'ultimi dati erano inconsistenti e senza alcuni valori, 
sono stati utilizzati per definire il ranking ELO e la correlazione.

A conclusione del progetto, le features che sono state utilizzate sono 24: Home team, away team, 
home team elo score, away team elo score, home team days since last match, 
away team days since last match, home team win rate, away team win rate, 
away team draw rate, home team draw rate, Last 7 home team win rate, 
Last 7 home team lose rate, last 7 home team draw rate, Last 7 away team win rate, 
Last 7 away team lose rate, last 7 away team draw rate, Last 12 home team win rate, 
Last 12 home team lose rate, last 12 home team draw rate, Last 12 away team win rate, 
Last 12 away team lose rate, last 12 away team draw rate, last 5 home team home win 
and last 5 away team away win.

I dati sulla stagione corrente (penso 2018/2019) sono stati soggetti ad augmentazione,
poiché per ogni partita sono state calcolate quante vittorie, sconfitte e pareggi 
le squadre hanno conseguito fino alla partita in questione, così come la loro serie di vittorie o sconfitte.

% \subsection{Selezione delle features}
